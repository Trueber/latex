\documentclass[mathsf,aspectratio=169]{beamer}
\usepackage[utf8]{inputenc}
\usepackage[T1]{fontenc}
%\usepackage{helvet}
\usepackage{arev}
\usepackage{tikz}

\title{Titel}
\subtitle{Untertitel}
\date[ISPN ’80]{Zusatzinfo wie z.B. \\ Anlass}
%\date{\strut}
\author[Heck]{Horst Heck \\ MNG}
\institute{Technik und Informatik}
% Die Titelgrafik ist 4cm auf 9.7cm gross!
\titlegraphic{julia_escher.jpg}

%Titelseite mit Grafik
\usetheme[titelgraphik]{bfhbeamer}
% Titelseite ohne Grafik, Hintergrund grau
%\usetheme{bfhbeamer}


\begin{document}

	\begin{frame}
		\titlepage
	\end{frame}
	
	\begin{frame}[t]
		\frametitle{Frametitle}
		\framesubtitle{Untertitel}
		Testing
		\begin{itemize}
			\item Testet
			\begin{itemize}
				\item \textcolor{red}{Nested} 
				\pause
				\item Nested
			\end{itemize}
			\item Test $\int f(x) \partial_x$
		\end{itemize}
	\end{frame}
	
	\part{Kapiteltrennseite}
	\frame{\partpage}
	
	\section{Abschnittstrennseite}
		\begin{frame}
			\sectionpage
		\end{frame}
		\begin{frame}
			\frametitle{Test 2}
			gerahmter Block:
			\begin{block}{Blocktitel}
				        Blocktext
			\end{block}
			\begin{alertblock}{Alert-Blocktitel}
				        Blocktext
			\end{alertblock}
			Oder man macht es mit ``rahmen'' (dann gibt es aber keinen Abstand)
			\begin{rahmen}[frametitle={Rahmentitel}]
				Rahmeninhalt
			\end{rahmen}
		\end{frame}
\end{document}
